% Options for packages loaded elsewhere
\PassOptionsToPackage{unicode}{hyperref}
\PassOptionsToPackage{hyphens}{url}
%
\documentclass[
]{article}
\usepackage{amsmath,amssymb}
\usepackage{iftex}
\ifPDFTeX
  \usepackage[T1]{fontenc}
  \usepackage[utf8]{inputenc}
  \usepackage{textcomp} % provide euro and other symbols
\else % if luatex or xetex
  \usepackage{unicode-math} % this also loads fontspec
  \defaultfontfeatures{Scale=MatchLowercase}
  \defaultfontfeatures[\rmfamily]{Ligatures=TeX,Scale=1}
\fi
\usepackage{lmodern}
\ifPDFTeX\else
  % xetex/luatex font selection
\fi
% Use upquote if available, for straight quotes in verbatim environments
\IfFileExists{upquote.sty}{\usepackage{upquote}}{}
\IfFileExists{microtype.sty}{% use microtype if available
  \usepackage[]{microtype}
  \UseMicrotypeSet[protrusion]{basicmath} % disable protrusion for tt fonts
}{}
\makeatletter
\@ifundefined{KOMAClassName}{% if non-KOMA class
  \IfFileExists{parskip.sty}{%
    \usepackage{parskip}
  }{% else
    \setlength{\parindent}{0pt}
    \setlength{\parskip}{6pt plus 2pt minus 1pt}}
}{% if KOMA class
  \KOMAoptions{parskip=half}}
\makeatother
\usepackage{xcolor}
\usepackage[margin=1in]{geometry}
\usepackage{color}
\usepackage{fancyvrb}
\newcommand{\VerbBar}{|}
\newcommand{\VERB}{\Verb[commandchars=\\\{\}]}
\DefineVerbatimEnvironment{Highlighting}{Verbatim}{commandchars=\\\{\}}
% Add ',fontsize=\small' for more characters per line
\usepackage{framed}
\definecolor{shadecolor}{RGB}{248,248,248}
\newenvironment{Shaded}{\begin{snugshade}}{\end{snugshade}}
\newcommand{\AlertTok}[1]{\textcolor[rgb]{0.94,0.16,0.16}{#1}}
\newcommand{\AnnotationTok}[1]{\textcolor[rgb]{0.56,0.35,0.01}{\textbf{\textit{#1}}}}
\newcommand{\AttributeTok}[1]{\textcolor[rgb]{0.13,0.29,0.53}{#1}}
\newcommand{\BaseNTok}[1]{\textcolor[rgb]{0.00,0.00,0.81}{#1}}
\newcommand{\BuiltInTok}[1]{#1}
\newcommand{\CharTok}[1]{\textcolor[rgb]{0.31,0.60,0.02}{#1}}
\newcommand{\CommentTok}[1]{\textcolor[rgb]{0.56,0.35,0.01}{\textit{#1}}}
\newcommand{\CommentVarTok}[1]{\textcolor[rgb]{0.56,0.35,0.01}{\textbf{\textit{#1}}}}
\newcommand{\ConstantTok}[1]{\textcolor[rgb]{0.56,0.35,0.01}{#1}}
\newcommand{\ControlFlowTok}[1]{\textcolor[rgb]{0.13,0.29,0.53}{\textbf{#1}}}
\newcommand{\DataTypeTok}[1]{\textcolor[rgb]{0.13,0.29,0.53}{#1}}
\newcommand{\DecValTok}[1]{\textcolor[rgb]{0.00,0.00,0.81}{#1}}
\newcommand{\DocumentationTok}[1]{\textcolor[rgb]{0.56,0.35,0.01}{\textbf{\textit{#1}}}}
\newcommand{\ErrorTok}[1]{\textcolor[rgb]{0.64,0.00,0.00}{\textbf{#1}}}
\newcommand{\ExtensionTok}[1]{#1}
\newcommand{\FloatTok}[1]{\textcolor[rgb]{0.00,0.00,0.81}{#1}}
\newcommand{\FunctionTok}[1]{\textcolor[rgb]{0.13,0.29,0.53}{\textbf{#1}}}
\newcommand{\ImportTok}[1]{#1}
\newcommand{\InformationTok}[1]{\textcolor[rgb]{0.56,0.35,0.01}{\textbf{\textit{#1}}}}
\newcommand{\KeywordTok}[1]{\textcolor[rgb]{0.13,0.29,0.53}{\textbf{#1}}}
\newcommand{\NormalTok}[1]{#1}
\newcommand{\OperatorTok}[1]{\textcolor[rgb]{0.81,0.36,0.00}{\textbf{#1}}}
\newcommand{\OtherTok}[1]{\textcolor[rgb]{0.56,0.35,0.01}{#1}}
\newcommand{\PreprocessorTok}[1]{\textcolor[rgb]{0.56,0.35,0.01}{\textit{#1}}}
\newcommand{\RegionMarkerTok}[1]{#1}
\newcommand{\SpecialCharTok}[1]{\textcolor[rgb]{0.81,0.36,0.00}{\textbf{#1}}}
\newcommand{\SpecialStringTok}[1]{\textcolor[rgb]{0.31,0.60,0.02}{#1}}
\newcommand{\StringTok}[1]{\textcolor[rgb]{0.31,0.60,0.02}{#1}}
\newcommand{\VariableTok}[1]{\textcolor[rgb]{0.00,0.00,0.00}{#1}}
\newcommand{\VerbatimStringTok}[1]{\textcolor[rgb]{0.31,0.60,0.02}{#1}}
\newcommand{\WarningTok}[1]{\textcolor[rgb]{0.56,0.35,0.01}{\textbf{\textit{#1}}}}
\usepackage{graphicx}
\makeatletter
\def\maxwidth{\ifdim\Gin@nat@width>\linewidth\linewidth\else\Gin@nat@width\fi}
\def\maxheight{\ifdim\Gin@nat@height>\textheight\textheight\else\Gin@nat@height\fi}
\makeatother
% Scale images if necessary, so that they will not overflow the page
% margins by default, and it is still possible to overwrite the defaults
% using explicit options in \includegraphics[width, height, ...]{}
\setkeys{Gin}{width=\maxwidth,height=\maxheight,keepaspectratio}
% Set default figure placement to htbp
\makeatletter
\def\fps@figure{htbp}
\makeatother
\setlength{\emergencystretch}{3em} % prevent overfull lines
\providecommand{\tightlist}{%
  \setlength{\itemsep}{0pt}\setlength{\parskip}{0pt}}
\setcounter{secnumdepth}{-\maxdimen} % remove section numbering
\ifLuaTeX
  \usepackage{selnolig}  % disable illegal ligatures
\fi
\IfFileExists{bookmark.sty}{\usepackage{bookmark}}{\usepackage{hyperref}}
\IfFileExists{xurl.sty}{\usepackage{xurl}}{} % add URL line breaks if available
\urlstyle{same}
\hypersetup{
  pdftitle={Final Paper Study 3 Analysis},
  hidelinks,
  pdfcreator={LaTeX via pandoc}}

\title{Final Paper Study 3 Analysis}
\author{}
\date{\vspace{-2.5em}2023-12-11}

\begin{document}
\maketitle

\hypertarget{setup}{%
\subsection{Setup}\label{setup}}

\begin{Shaded}
\begin{Highlighting}[]
\FunctionTok{library}\NormalTok{(tidyverse)}
\end{Highlighting}
\end{Shaded}

\begin{verbatim}
## -- Attaching packages --------------------------------------- tidyverse 1.3.2 --
## v ggplot2 3.4.0      v purrr   1.0.1 
## v tibble  3.1.8      v dplyr   1.0.10
## v tidyr   1.2.1      v stringr 1.5.0 
## v readr   2.1.3      v forcats 0.5.2 
## -- Conflicts ------------------------------------------ tidyverse_conflicts() --
## x dplyr::filter() masks stats::filter()
## x dplyr::lag()    masks stats::lag()
\end{verbatim}

\begin{Shaded}
\begin{Highlighting}[]
\FunctionTok{library}\NormalTok{(emmeans)}
\FunctionTok{library}\NormalTok{(psych)}
\end{Highlighting}
\end{Shaded}

\begin{verbatim}
## 
## Attaching package: 'psych'
## 
## The following objects are masked from 'package:ggplot2':
## 
##     %+%, alpha
\end{verbatim}

\begin{Shaded}
\begin{Highlighting}[]
\FunctionTok{library}\NormalTok{(multcomp)}
\end{Highlighting}
\end{Shaded}

\begin{verbatim}
## Loading required package: mvtnorm
## Loading required package: survival
## Loading required package: TH.data
## Loading required package: MASS
## 
## Attaching package: 'MASS'
## 
## The following object is masked from 'package:dplyr':
## 
##     select
## 
## 
## Attaching package: 'TH.data'
## 
## The following object is masked from 'package:MASS':
## 
##     geyser
\end{verbatim}

\hypertarget{data-creation}{%
\subsection{Data Creation}\label{data-creation}}

\textbf{reminder for future Rowan and anyone else that this is
completely fake data that should never actually be used for a real
paper!!}

First, I am simulating the data for each of my four conditions. My goal
was to create four groups of 40 participants (total N = 160) where the
overall reaction time was 450 ms with a standard deviation of 50 ms. In
line with my hypothesis, I would like the richness group to have a
significantly higher reaction time (i.e., about 2 standard deviations or
100 ms higher) than any other group. Here are the benchmarks I set for
means and SDs for each of the four groups:

Control: 425 ms (50) Happiness: 450 ms (50) Meaning: 475 ms (50) PR: 575
ms (50)

\begin{Shaded}
\begin{Highlighting}[]
\CommentTok{\# Set seed for reproducibility}
\FunctionTok{set.seed}\NormalTok{(}\DecValTok{57}\NormalTok{)}

\CommentTok{\# Function to generate simulated data for each of the four conditions}
\NormalTok{generate\_simulated\_data }\OtherTok{\textless{}{-}} \ControlFlowTok{function}\NormalTok{(condition\_name, num\_participants, mean\_reaction\_time, std\_deviation) \{}
  \FunctionTok{tibble}\NormalTok{(}
    \AttributeTok{condition =}\NormalTok{ condition\_name,}
    \AttributeTok{participant\_id =} \DecValTok{1}\SpecialCharTok{:}\NormalTok{num\_participants,}
    \AttributeTok{reaction\_time =} \FunctionTok{rnorm}\NormalTok{(num\_participants, }\AttributeTok{mean =}\NormalTok{ mean\_reaction\_time, }\AttributeTok{sd =}\NormalTok{ std\_deviation)}
\NormalTok{  )}
\NormalTok{\}}

\CommentTok{\# Parameters for each condition (making sure there are 160 unique cases created)}
\NormalTok{control\_data }\OtherTok{\textless{}{-}} \FunctionTok{generate\_simulated\_data}\NormalTok{(}\StringTok{"Control"}\NormalTok{, }\DecValTok{40}\NormalTok{, }\DecValTok{425}\NormalTok{, }\DecValTok{50}\NormalTok{)}
\NormalTok{happiness\_data }\OtherTok{\textless{}{-}} \FunctionTok{generate\_simulated\_data}\NormalTok{(}\StringTok{"Happiness"}\NormalTok{, }\DecValTok{40}\NormalTok{, }\DecValTok{450}\NormalTok{, }\DecValTok{54}\NormalTok{) }\SpecialCharTok{\%\textgreater{}\%}
  \FunctionTok{mutate}\NormalTok{(}\AttributeTok{participant\_id =}\NormalTok{ participant\_id }\SpecialCharTok{+} \FunctionTok{max}\NormalTok{(control\_data}\SpecialCharTok{$}\NormalTok{participant\_id))}
\NormalTok{meaning\_data }\OtherTok{\textless{}{-}} \FunctionTok{generate\_simulated\_data}\NormalTok{(}\StringTok{"Meaning"}\NormalTok{, }\DecValTok{40}\NormalTok{, }\DecValTok{475}\NormalTok{, }\DecValTok{52}\NormalTok{) }\SpecialCharTok{\%\textgreater{}\%}
  \FunctionTok{mutate}\NormalTok{(}\AttributeTok{participant\_id =}\NormalTok{ participant\_id }\SpecialCharTok{+} \FunctionTok{max}\NormalTok{(happiness\_data}\SpecialCharTok{$}\NormalTok{participant\_id))}
\NormalTok{richness\_data }\OtherTok{\textless{}{-}} \FunctionTok{generate\_simulated\_data}\NormalTok{(}\StringTok{"Psychological Richness"}\NormalTok{, }\DecValTok{40}\NormalTok{, }\DecValTok{575}\NormalTok{, }\DecValTok{55}\NormalTok{) }\SpecialCharTok{\%\textgreater{}\%}
  \FunctionTok{mutate}\NormalTok{(}\AttributeTok{participant\_id =}\NormalTok{ participant\_id }\SpecialCharTok{+} \FunctionTok{max}\NormalTok{(meaning\_data}\SpecialCharTok{$}\NormalTok{participant\_id))}


\CommentTok{\# Combine data from all four conditions}
\NormalTok{combined\_data }\OtherTok{\textless{}{-}} \FunctionTok{bind\_rows}\NormalTok{(control\_data, happiness\_data, meaning\_data, richness\_data)}

\CommentTok{\# Convert the "condition" variable to a factor with specified levels}
\NormalTok{combined\_data}\SpecialCharTok{$}\NormalTok{condition }\OtherTok{\textless{}{-}} \FunctionTok{factor}\NormalTok{(combined\_data}\SpecialCharTok{$}\NormalTok{condition, }\AttributeTok{levels =} \FunctionTok{c}\NormalTok{(}\StringTok{"Control"}\NormalTok{, }\StringTok{"Happiness"}\NormalTok{, }\StringTok{"Meaning"}\NormalTok{, }\StringTok{"Psychological Richness"}\NormalTok{))}

\CommentTok{\# Print the combined data}
\FunctionTok{print}\NormalTok{(combined\_data)}
\end{Highlighting}
\end{Shaded}

\begin{verbatim}
## # A tibble: 160 x 3
##    condition participant_id reaction_time
##    <fct>              <int>         <dbl>
##  1 Control                1          390.
##  2 Control                2          337.
##  3 Control                3          456.
##  4 Control                4          526.
##  5 Control                5          432.
##  6 Control                6          506.
##  7 Control                7          495.
##  8 Control                8          381.
##  9 Control                9          374.
## 10 Control               10          473.
## # ... with 150 more rows
\end{verbatim}

\hypertarget{data-analysis}{%
\subsection{Data Analysis}\label{data-analysis}}

Now I will run my analysis: a planned contrast (with coefficients
control = 0, happiness = -1, meaning = -1, and psychological richness =
2) testing my hypothesis that participants in the psychological richness
condition will have higher (i.e., slower) reaction times than any other
well-being condition.

(First checking that my descriptives generally line up with what I
wanted them to be)

\begin{Shaded}
\begin{Highlighting}[]
\FunctionTok{describeBy}\NormalTok{(combined\_data}\SpecialCharTok{$}\NormalTok{reaction\_time, combined\_data}\SpecialCharTok{$}\NormalTok{condition, }\AttributeTok{mat=}\ConstantTok{TRUE}\NormalTok{)}
\end{Highlighting}
\end{Shaded}

\begin{verbatim}
##     item                 group1 vars  n     mean       sd   median  trimmed
## X11    1                Control    1 40 424.5502 67.16756 428.1411 425.2686
## X12    2              Happiness    1 40 458.4398 48.38017 456.3601 458.5505
## X13    3                Meaning    1 40 473.5943 49.58174 473.3858 470.7418
## X14    4 Psychological Richness    1 40 572.2861 65.24677 578.4745 572.2576
##          mad      min      max    range        skew   kurtosis        se
## X11 72.84095 288.1843 544.0270 255.8428 -0.03207518 -0.9137963 10.620123
## X12 51.47009 350.0421 550.6286 200.5865 -0.03213957 -0.7377662  7.649576
## X13 50.23059 382.8008 596.7871 213.9863  0.42461303 -0.2068782  7.839562
## X14 55.69665 445.5886 727.6205 282.0319 -0.02990278 -0.4715321 10.316420
\end{verbatim}

Running an omnibus ANOVA to start:

\begin{Shaded}
\begin{Highlighting}[]
\NormalTok{model }\OtherTok{\textless{}{-}} \FunctionTok{aov}\NormalTok{(reaction\_time }\SpecialCharTok{\textasciitilde{}}\NormalTok{ condition, }\AttributeTok{data =}\NormalTok{ combined\_data)}
\FunctionTok{summary}\NormalTok{(model)}
\end{Highlighting}
\end{Shaded}

\begin{verbatim}
##              Df Sum Sq Mean Sq F value Pr(>F)    
## condition     3 483105  161035   47.48 <2e-16 ***
## Residuals   156 529137    3392                   
## ---
## Signif. codes:  0 '***' 0.001 '**' 0.01 '*' 0.05 '.' 0.1 ' ' 1
\end{verbatim}

Big surprise, it is incredibly significant :)

Now I will run my planned contrast:

\begin{Shaded}
\begin{Highlighting}[]
\CommentTok{\# verifying my IV levels are what I think they are (and that my turning "condition" into a factor earlier worked)}

\FunctionTok{levels}\NormalTok{(combined\_data}\SpecialCharTok{$}\NormalTok{condition)}
\end{Highlighting}
\end{Shaded}

\begin{verbatim}
## [1] "Control"                "Happiness"              "Meaning"               
## [4] "Psychological Richness"
\end{verbatim}

\begin{Shaded}
\begin{Highlighting}[]
\CommentTok{\# here is a contrast verifying that the happiness, meaning, and richness conditions have a higher average reaction time than the control group}
\NormalTok{c1 }\OtherTok{\textless{}{-}} \FunctionTok{c}\NormalTok{(}\SpecialCharTok{{-}}\DecValTok{3}\NormalTok{, }\DecValTok{1}\NormalTok{, }\DecValTok{1}\NormalTok{, }\DecValTok{1}\NormalTok{)}
\CommentTok{\# here is my hypothesized contrast:}
\NormalTok{c2 }\OtherTok{\textless{}{-}} \FunctionTok{c}\NormalTok{(}\DecValTok{0}\NormalTok{,}\SpecialCharTok{{-}}\DecValTok{1}\NormalTok{,}\SpecialCharTok{{-}}\DecValTok{1}\NormalTok{,}\DecValTok{2}\NormalTok{)}

\CommentTok{\# combining the above 2 lines into a matrix:}
\NormalTok{mat }\OtherTok{\textless{}{-}} \FunctionTok{cbind}\NormalTok{(c1,c2)}

\CommentTok{\# defining the contrasts matrix:}
\FunctionTok{contrasts}\NormalTok{(combined\_data}\SpecialCharTok{$}\NormalTok{condition) }\OtherTok{\textless{}{-}}\NormalTok{ mat}

\CommentTok{\# getting results:}
\NormalTok{model1 }\OtherTok{\textless{}{-}} \FunctionTok{aov}\NormalTok{(reaction\_time }\SpecialCharTok{\textasciitilde{}}\NormalTok{ condition, }\AttributeTok{data =}\NormalTok{ combined\_data)}
\FunctionTok{summary}\NormalTok{(model1)}
\end{Highlighting}
\end{Shaded}

\begin{verbatim}
##              Df Sum Sq Mean Sq F value Pr(>F)    
## condition     3 483105  161035   47.48 <2e-16 ***
## Residuals   156 529137    3392                   
## ---
## Signif. codes:  0 '***' 0.001 '**' 0.01 '*' 0.05 '.' 0.1 ' ' 1
\end{verbatim}

\begin{Shaded}
\begin{Highlighting}[]
\FunctionTok{summary.aov}\NormalTok{(model1, }\AttributeTok{split=}\FunctionTok{list}\NormalTok{(}\AttributeTok{condition=}\FunctionTok{list}\NormalTok{(}\StringTok{"Control vs. Well{-}Being Conditions"} \OtherTok{=} \DecValTok{1}\NormalTok{, }\StringTok{"Psychological Richness vs. Happiness and Meaning"}\OtherTok{=}\DecValTok{2}\NormalTok{))) }
\end{Highlighting}
\end{Shaded}

\begin{verbatim}
##                                                                Df Sum Sq
## condition                                                       3 483105
##   condition: Control vs. Well-Being Conditions                  1 177362
##   condition: Psychological Richness vs. Happiness and Meaning   1 301150
## Residuals                                                     156 529137
##                                                               Mean Sq F value
## condition                                                      161035   47.48
##   condition: Control vs. Well-Being Conditions                 177362   52.29
##   condition: Psychological Richness vs. Happiness and Meaning  301150   88.78
## Residuals                                                        3392        
##                                                                 Pr(>F)    
## condition                                                      < 2e-16 ***
##   condition: Control vs. Well-Being Conditions                2.02e-11 ***
##   condition: Psychological Richness vs. Happiness and Meaning  < 2e-16 ***
## Residuals                                                                 
## ---
## Signif. codes:  0 '***' 0.001 '**' 0.01 '*' 0.05 '.' 0.1 ' ' 1
\end{verbatim}

\begin{Shaded}
\begin{Highlighting}[]
\DocumentationTok{\#\#\# (Another method of getting the contrasts: getting the actual contrast values instead of just how they fit into the ANOVA)}


\CommentTok{\# Define the contrasts}
\NormalTok{contrasts\_matrix }\OtherTok{\textless{}{-}} \FunctionTok{matrix}\NormalTok{(}\FunctionTok{c}\NormalTok{(}\SpecialCharTok{{-}}\DecValTok{3}\NormalTok{, }\DecValTok{1}\NormalTok{, }\DecValTok{1}\NormalTok{, }\DecValTok{1}\NormalTok{, }\DecValTok{0}\NormalTok{, }\SpecialCharTok{{-}}\DecValTok{1}\NormalTok{, }\SpecialCharTok{{-}}\DecValTok{1}\NormalTok{, }\DecValTok{2}\NormalTok{), }\AttributeTok{ncol =} \DecValTok{4}\NormalTok{, }\AttributeTok{byrow =} \ConstantTok{TRUE}\NormalTok{)}

\CommentTok{\# Specify the factor variable and the planned contrasts}
\NormalTok{contrasts\_condition }\OtherTok{\textless{}{-}} \FunctionTok{glht}\NormalTok{(model1, }\AttributeTok{linfct =} \FunctionTok{mcp}\NormalTok{(}\AttributeTok{condition =}\NormalTok{ contrasts\_matrix))}

\CommentTok{\# Summary of planned contrasts}
\FunctionTok{summary}\NormalTok{(contrasts\_condition)}
\end{Highlighting}
\end{Shaded}

\begin{verbatim}
## 
##   Simultaneous Tests for General Linear Hypotheses
## 
## Multiple Comparisons of Means: User-defined Contrasts
## 
## 
## Fit: aov(formula = reaction_time ~ condition, data = combined_data)
## 
## Linear Hypotheses:
##        Estimate Std. Error t value Pr(>|t|)    
## 1 == 0   230.67      31.90   7.231   <1e-10 ***
## 2 == 0   212.54      22.56   9.423   <1e-10 ***
## ---
## Signif. codes:  0 '***' 0.001 '**' 0.01 '*' 0.05 '.' 0.1 ' ' 1
## (Adjusted p values reported -- single-step method)
\end{verbatim}

\end{document}
